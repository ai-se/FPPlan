\begin{abstract}

    The current generation of software analytics tools are mostly prediction algorithms (e.g. support vector machines, naive
    bayes, logistic regression, etc). While prediction is useful, after prediction comes {\em planning} about what actions to take in order to
    improve quality. This research seeks methods that generate demonstrably useful guidance on ``what to do'' within the context of a
    specific software project. Specifically, we propose XTREE (for within-project planning) and BELLTREE (for cross-project planning) to
    generating plans that can improve software quality. Each such plan has the property that, if followed, it reduces the expected number of future
    defect reports.
    To find this expected number,   planning was first applied to data from release $x$. Next,  we looked for change  in release $x+1$
    that conformed to our plans. 
    This procedure was applied using a range of planners from the literature, as well as  XTREE. 
    In 10 open-source JAVA systems, several hundreds of defects
    were reduced in sections of the code that conformed to XTREE's  plans.
    Further, when compared to other planners,
    XTREE's plans were found to be easier to implement (since they were shorter) and more effective at reducing the expected number of defects.
    % across projects, which is particularly useful when there are no historical logs available for a particular project to generate plans from.
    \keywords{Data Mining, Actionable Analytics, Planning, bellwethers, defect prediction.}
    \end{abstract}
    