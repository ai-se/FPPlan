\section{Discussion}
\label{sect:discuss}
When discussing these results with colleagues, we are often asked the following questions.

\textit{1. Why use automatic methods to find quality plans? Why not just use domain knowledge; e.g. human expert intuition?} Recent research has documented the wide variety of conflicting opinions among software developers, even those working within the same project. According to Passos et al.~\citep{passos11}, developers often assume that the lessons they learn from a few past projects are general to all their future projects. They comment, ``past experiences were taken into account without
much consideration for their context''. Jorgensen and Gruschke~\citep{jorgensen09} offer a similar warning. They report that the supposed software engineering ``gurus'' rarely use lessons from past projects to improve their future reasoning and that such poor past advice can be detrimental to new projects~\citep{jorgensen09}. Other studies have shown some widely-held views are now questionable given new evidence. Devanbu et al. examined responses from 564 Microsoft software developers from around the world. They comment programmer beliefs can vary with each project, but do not necessarily correspond with actual evidence in that project~\citep{prem16}. Given the diversity of opinions seen among humans, it seems wise to explore automatic oracles for planning.

\textit{2. Does using BELLTREE guarantee that software managers will never have to change their plans?} No. Software managers should evolve their policies when the evolving circumstances require such an update. But how to know when to retain current policies or when to switch to new ones? Bellwether method can answer this question.

Specifically, we advocate continually retesting the bellwether's status against other data sets within the community. If a new bellwether is found, then it is time for the community to accept very different policies. Otherwise, it is valid for managers to ignore most the new data arriving into that community.

